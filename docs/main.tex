\documentclass{article}


\usepackage{PRIMEarxiv}

\usepackage[utf8]{inputenc} % allow utf-8 input
\usepackage[T1]{fontenc}    % use 8-bit T1 fonts
\usepackage{hyperref}       % hyperlinks
\usepackage{url}            % simple URL typesetting
\usepackage{booktabs}       % professional-quality tables
\usepackage{amsfonts}       % blackboard math symbols
\usepackage{nicefrac}       % compact symbols for 1/2, etc.
\usepackage{microtype}      % microtypography
\usepackage{lipsum}
\usepackage{graphicx}
\usepackage{booktabs, multirow} 
\graphicspath{{media/}}     % organize your images and other figures under media/ folder


%% Title
\title{We Didn't Start The Fire
%%%% Cite as
% %%%% Update your official citation here when published 
% \thanks{\textit{\underline{Citation}}: 
% \textbf{Authors. Title. Pages.... DOI:000000/11111.}} 
}

\author{
  Ivanina Ivanova, Abhay Mathur \\
  Institut Polytechnique de Paris \\
  \texttt{\{abhay.mathur, ivanina.ivanova\}@ip-paris.fr} \\
  %% examples of more authors
}


\begin{document}
\maketitle

\begin{abstract}
  \lipsum[1]
\end{abstract}

% keywords can be removed
\keywords{First keyword \and Second keyword \and More}

\section{Introduction: Problem Statement}
This project aims to develop a fire detection system using a designated
dataset. The dataset consists of three subsets: a training set, a validation
set, and a test set. Specific constraints and guidelines must be strictly
followed to ensure compliance with the project requirements.

\subsection{Dataset Access and Constraints}

The dataset is available for download at
\url{https://www.kaggle.com/datasets/abdelghaniaaba/wildfire-prediction-dataset/code}.
It comprises a training set, a validation set, and a test set. A critical
restriction is imposed on the training set: its labels are inaccessible. Any
direct utilization of annotations from the training set will lead to
disqualification.

\subsection{Dataset Partitioning}

To facilitate model training, the original validation set must be partitioned
into a newly defined validation set and a new training set. The original
training set may be leveraged in a creative manner; however, its labels must
not be employed under any circumstances.

\subsection{Model Development}

A deep neural network (DNN) will be trained utilizing the newly defined
training and validation sets. Various methodologies and supplementary resources
may be incorporated to enhance model performance, provided that all constraints
related to annotation usage are rigorously upheld.

\section{Method}\label{sec:method}

\subsection{Dataset Analysis}
\dots

\subsection{Proposition 1: Using Available Labeled Data}
\subsubsection{Naive Coordinates Classifier}
\dots

\subsubsection{ResNet Classifier}
\dots

\subsection{Self-Supervision (Learning from Unlabeled Data)}

\subsection{SimCLR}
Just wanted to cite this paper~\cite{simclr}.

\begin{table}[!htp]\centering
  \caption{Comparisons}\label{tab: }
  \scriptsize
  \begin{tabular}{lrrrr}\toprule
                                               & \textbf{Model}  & \textbf{Accuracy} & \textbf{F1 Score} \\\cmidrule{2-4}
    \multirow{2}{*}{\textbf{No pre-training}}  & Coords Only     & 0.855             & 0.873             \\ %\cmidrule{2-4}
                                               & ResNet50        & 0.985             & 0.986             \\\cmidrule{1-4}
    \multirow{3}{*}{\textbf{Self-Supervision}} & SimCLR+ResNet50 & ---               & ---               \\ %\cmidrule{2-4}
                                               & VAE             & ---               & ---               \\ %\cmidrule{2-4}
                                               & Something Else  & ---               & ---               \\\midrule
    \bottomrule
  \end{tabular}
\end{table}

\section{Conclusion}
Your conclusion here

%Bibliography
\bibliographystyle{unsrt}
\bibliography{references}

\end{document}
